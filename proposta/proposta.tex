%        File: planoic.tex
%     Created: Fri Apr 18 01:00 PM 2014 B
% Last Change: Fri Apr 18 01:00 PM 2014 B
%
\documentclass[12pt, a4paper, oneside]{article}

\usepackage{amsfonts}
\usepackage[]{amsmath}
\usepackage{amssymb}
\usepackage[brazil]{babel}
\usepackage[left=20mm,right=20mm,top=25mm,bottom=25mm]{geometry}
\usepackage[]{graphicx}
\usepackage[utf8]{inputenc}
\usepackage{multirow}
\usepackage{paralist}
\usepackage{setspace}
\usepackage[compact]{titlesec}
\usepackage[]{url}
\usepackage{xspace}
\usepackage{mathpazo}
\usepackage{txfonts}

%\usepackage[]{biblatex}

\titleformat{\section}
  {\Large\bfseries}{\thesection.}{.5em}{}[\hrule\bigskip]
\titleformat{\subsection}
  {\bfseries}{\thesection.}{.5em}{}
%\renewcommand{\baselinestretch}{1.5} 
%\renewcommand{\rmdefault}{phv} % Arial
%\renewcommand{\sfdefault}{phv} % Arial
\setlength{\parskip}{6pt}

\newcommand{\A}{\ensuremath{\mathtt{A}}\xspace}
\newcommand{\C}{\ensuremath{\mathtt{C}}\xspace}
\newcommand{\G}{\ensuremath{\mathtt{G}}\xspace}
\newcommand{\T}{\ensuremath{\mathtt{T}}\xspace}

\newcommand{\str}[1]{\ensuremath{\mathtt{#1}}\xspace}
\newcommand{\strset}[1]{\ensuremath{\mathcal{#1}}\xspace}
\newcommand{\ssS}{\strset{S}}
\newcommand{\seq}[1]{\ensuremath{\mathtt{#1}}\xspace}

\newcommand{\rank}{\ensuremath{\mathrm{rank}}\xspace}
\newcommand{\select}{\ensuremath{\mathrm{select}}\xspace}
\newcommand{\BWT}{\ensuremath{\mathrm{BWT}}\xspace}

\newcommand{\X}{\ensuremath{\medbullet}\xspace}
\newcommand{\x}{\ensuremath{\medcirc}\xspace}


%\bibliography{projeto}


\begin{document}
%\onehalfspacing

%\maketitle

\thispagestyle{empty}
\begin{center}
\Large
Universidade Federal de Pernambuco\\
Centro de Informática\\
Graduação em Ciência da Computação

\vfill

{\huge \bfseries Um algoritmo eficiente em tempo e espaço para o problema do ancestral comum mais profundo em árvores e grafos acíclicos }
\\
\medskip
{\bfseries\itshape Proposta de Trabalho de Graduação}

\vfill

\bigskip

	\begin{tabular}{r p{95mm}}
	\textbf{Aluno: } & Daniel Cândido Cauás\newline(\texttt{dcc4@cin.ufpe.br}) \\ 
\textbf{Orientador: } & Paulo Gustavo Soares da Fonseca \newline(\texttt{paguso@cin.ufpe.br})
\\
	\textbf{Área: } & Teoria da Computação / Algoritmos
\end{tabular}

	\vspace{3cm}
Recife, Março de 2019 
\end{center}

\clearpage 
\thispagestyle{empty}
\section*{Resumo}
Um problema algorítmico recorrente em áreas diversas como Engenharia de Software ou Biologia Computacional, é o de encontrar o ancestral comum mais profundo (\textit{Lowest Common Ancestor}---LCA) em árvores e/ou grafos dirigidos. As principais soluções existentes consistem em pré-processar a árvore para facilitar a consulta, e portanto, são avaliadas em termos do tempo e espaço de pré-processamento e de consulta. O objetivo de desse trabalho é propor uma solução simples para consultas de LCAs (queries) sucessivas e avaliar o seu desempenho em termos teóricos e práticos com respeito ao tempo de pré-processamento, consulta e uso de memória, em comparação a algumas soluções consagradas encontradas na literatura.

\clearpage
\setcounter{page}{1}
\section*{Introdução}

Em ciência da computação, as árvores são estruturas ubíquas, utilizadas nas mais variadas aplicações, devido à sua versatilidade. As árvores formam a base de implementação de diversas estruturas de dados de várias linguagens de programação. Na área de Engenharia de Software, por exemplo, são usadas para modelar a herança em linguagens orientadas a objetos. Na área de Biologia Computacional, são usadas para representar árvores evolutivas. Nessas e em outras aplicações, um problema recorrente é o de determinar o ancestral comum mais profundo entre dois nós, em Inglês \textit{Lowest common ancestor}---LCA. 

Quando a árvore de entrada é estática, uma solução frequentemente utilizada consistem em pré-processá-la para construir estruturas auxiliares que permitem consultas posteriores sem a necessidade de percorrer novamente a árvore. Bender e coautores \cite{Bender2005} apresentam algoritmos para o problema LCA baseados nessa abordagem, analisando-as em termos de tempo de pré-processamento e consulta. A solução ótima apresentada possui complexidade $\langle \Theta(n), \Theta(1)\rangle$, isto é, em tempo/espaço linear de pré-processamento e tempo constante de consulta.


\section*{Objetivos}

%Os algoritmos escritos para responder uma sequência de queries de LCAs consistem em duas fases, que são o algoritmo de pré-processamento da árvore, com a finalidade de gerar uma estrutura de dados que facilite a resposta do LCA, e a query propriamente dita, que é um algoritmo escrito sobre essa nova estrutura de dados, cujo fim é responder objetivamente o LCA da query.

Este trabalho tem por objetivo geral analisar o desempenho prático relativo de diferentes algoritmos para o problema LCA. Especificamente, além de algoritmos descritos na literatura, nós pretendemos propor otimizações práticas para estruturas de dados resultantes do  pré-processamento da árvore com o objetivo de obter uma solução eficiente em termos de espaço e tempo, permitindo ainda que as consultas sejam respondidas em tempo ótimo ou sub-ótimo na maioria dos casos, conforme verificado em estudos preliminares já realizados.


\section*{Metodologia}

O projeto será organizado nas atividades descritas abaixo e desenvolvidas conforme o cronograma a seguir.

\subsection*{T0. Preparação da proposta}

Nesta fase inicial, aluno e orientador farão uma série de reuniões para definição do problema e escopo do projeto. O orientador apresentará a bibliografia básica sobre o tema e os possíveis pontos a serem abordados, e ambos decidirão sobre aqueles a serem desenvolvidos com base no  interesse mútuo.

\subsection*{T1. Revisão e acompanhamento bibliográfico}

Essa tarefa será executada de maneira mais acentuada no início do projeto. A atividade consiste num estudo dirigido centrado no  material bibliográfico mais diretamente relacionado às estruturas de dados e algoritmos a serem implementados no projeto. Ao final dessa fase inicial, espera-se que o aluno possa manter-se atualizado e aprofundar-se em pontos específicos de maneira mais autônoma.


\subsection*{T2. Implementação das estruturas}

Esta tarefa corresponde ao principal componente de desenvolvimento do projeto. O aluno deverá, em interação com o orientador, estudar detalhadamente algoritmos e estruturas de dados relativos a árvores, LCA, sparse tables, entre outros, e desenvolver uma implementação de referência em nível de produção para os mesmos, em C++.


\subsection*{T3. Realização dos Testes}

Nesta tarefa, o aluno deverá fazer uma análise experimental comparativa do desempenho do método proposto e implementado relativamente a outras alternativas identificadas num levantamento inicial na tarefa T1. Pode ser necessário obter dados reais e produzir dados sintéticos que simulem uma situação limite de estresse para os métodos.


\subsection*{T4. Redação e revisão da monografia}

A monografia produzida deverá conter:
(1)~uma breve revisão bibliográfica do estado da arte, fruto da tarefa T1, (2)~descrição detalhada dos seus desenvolvimentos, incluindo a análise teórica dos algoritmos e estruturas propostas em termos de tempo e espaço, (3)~uma análise crítica com base em resultados experimentais da tarefa T3, e (4)~uma discussão com conclusões gerais sobre o projeto.


\subsection*{T5. Preparação da Apresentação}

Finalmente, o aluno preparará a apresentação da defesa do TG com o resumo dos desenvolvimentos e resultados obtidos.

\medskip

O aluno já possui familiaridade com a área, portanto poderá por-se rapidamente em desenvolvimento. O acompanhamento será feito pessoalmente através de reuniões semanais. Todo material desenvolvido é compartilhado entre orientador e aluno através de um repositório privado na plataforma GitHub. 


%\clearpage
\section*{Cronograma de atividades}


\begin{center}
	\begin{tabular}{| l || c | c | c | c | c | c | c | c | c | c | c | c | c | c | c |  c | c | c | c | }
		\hline
		& \multicolumn{2}{| c |}{Mar} & \multicolumn{4}{| c |}{Abr} & \multicolumn{4}{| c |}{Mai} & \multicolumn{4}{| c |}{Jun} & \multicolumn{2}{| c |}{Jul} \\\hline\hline
		Preparação da proposta & \X & \X & & & & & & & & & & & & & & \\\hline 
		Revisão bibliográfica$^\dagger$ & \X & \X & \x & \x & \x & \x & \x & \x & & & & & & & & \\\hline 
		Implementação das estruturas & & & \X & \X & \X & \X & \X & \X & \X & \X & \X & & & & & \\\hline 
		Realização dos testes$^\ddagger$ & & & & \x & \x & \x & \x & \x & \x  & \X & \X & \X & & & & \\\hline 
		Redação e revisão da monografia & & & & & & & & & & & \X & \X & \X & \X & & \\\hline 
		Preparação da apresentação & & & & & & & & & & & & & & & \X & \X \\\hline 
\hline
	\end{tabular}
\begin{minipage}{0.6\linewidth}
\noindent($\dagger$) \X = levantamento inicial, \x= aprofundamento\newline
\noindent($\ddagger$) \X= experimentos formais , \x = testes \textit{ad hoc} unitários\newline
\end{minipage}

\end{center}


\clearpage
\nocite{*}
\bibliographystyle{unsrt-etal}
\bibliography{proposta}
%\printbibliography

\clearpage
\section*{Possíveis Avaliadores}

\begin{enumerate}
\item Prof. Silvio Melo
\item Prof. Gustavo Carvalho
\end{enumerate}


%\clearpage
\section*{Assinaturas}

\vfill
\begin{center}
	Recife, 13 de Março de 2019

	\vspace{3cm}
	\rule{10cm}{.5pt}\\
	\textbf{Aluno:} Daniel Cândido Cauás\\

	\vspace{3cm}
	\rule{10cm}{.5pt}\\
	\textbf{Orientador:} Paulo Gustavo Soares da Fonseca\\
\end{center}
\vfill

\end{document}

